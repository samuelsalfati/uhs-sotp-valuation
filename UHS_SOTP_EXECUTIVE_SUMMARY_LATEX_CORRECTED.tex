\documentclass[12pt]{article}

% ---------- PACKAGES ----------
\usepackage[a4paper,margin=1in]{geometry}
\usepackage{graphicx}
\usepackage{booktabs}
\usepackage{array}
\usepackage{caption}
\usepackage{float}
\usepackage{setspace}
\usepackage{xcolor}
\usepackage{titlesec}
\usepackage{hyperref}
\usepackage{fancyhdr}
\usepackage{longtable}
\usepackage{amsmath}
\usepackage{multirow}

% ---------- SETTINGS ----------
\graphicspath{{./}}
\setstretch{1.15}
\hypersetup{
  colorlinks=true,
  urlcolor=blue,
  linkcolor=black,
  citecolor=black
}
\titleformat{\section}{\Large\bfseries}{\thesection}{1em}{}
\titleformat{\subsection}{\large\bfseries}{\thesubsection}{1em}{}
\pagestyle{fancy}
\fancyhf{}
\rhead{Ascendra Capital}
\lhead{UHS Valuation}
\cfoot{\thepage}

% ---------- DOCUMENT ----------
\begin{document}

% ---------- TITLE PAGE ----------
\begin{center}
    \vspace*{2em}
    \includegraphics[width=0.25\textwidth]{Assets/logo.png}\\[10pt]
    {\Huge \textbf{Universal Health Services (UHS)}}\\[6pt]
    {\Large \textbf{Comprehensive Valuation Analysis \& Investment Thesis}}\\[8pt]
    \textit{Prepared by Ascendra Capital}\\[12pt]
    \textbf{Date: November 7, 2025}\\[16pt]
\end{center}
\vfill
\begin{center}
\textbf{Current Price:} \$222.06\\
\textbf{Target Price (Base):} \$433\\
\textbf{Upside:} +95\%\\
\textbf{Recommendation:} \textbf{STRONG BUY}\\
\textbf{Weighted Average Fair Value:} \$392 / share
\end{center}
\clearpage

% ---------- EXECUTIVE SUMMARY ----------
\section*{Executive Summary}

Universal Health Services (NYSE: UHS) remains materially \textbf{undervalued}. Market pricing fails to capture the company's high-quality behavioral health operations (22.7\% margins) and substantial owned real estate portfolio (90.5\% facility ownership). Our updated analysis with properly normalized EBITDA supports a \textbf{base-case fair value of \$433 per share}, representing 95\% upside to the current price of \$222.06.

\subsection*{Key Findings}
\begin{enumerate}
  \item \textbf{Corrected SOTP Methodology:} Proper PropCo/OpCo split with normalized EBITDA yields \$433/share base case.
  \item \textbf{Behavioral Premium Justified:} 22.7\% EBITDA margins vs 13.5\% for Acute care support 9.0× OpCo multiple.
  \item \textbf{Real Estate Value:} \$18.4B in owned real estate valued at conservative 6.5\% cap rate.
  \item \textbf{DCF Validation:} 8.5\% WACC yields \$477/share.
  \item \textbf{Weighted Valuation:} Blended fair value \$392, range \$333-\$454.
\end{enumerate}

\subsection*{Valuation Summary}
\begin{table}[H]
\centering
\begin{tabular}{lcccc}
\toprule
Method & Weight & Base & Range & Notes \\
\midrule
SOTP (4-Part) & 30\% & \$433 & 399–471 & Corrected OpCo/PropCo split \\
DCF (10-Year) & 25\% & \$477 & 396–558 & 8.5\% WACC \\
LBO Model & 20\% & \$320 & 265–385 & Reverse-IRR constraint \\
Trading Comps & 10\% & \$236 & 193–280 & Peer multiple range \\
Precedent M\&A & 15\% & \$367 & 280–455 & Strategic transaction range \\
\midrule
\textbf{Weighted Average} & 100\% & \textbf{\$392} & 333–454 & Blended fair value \\
\bottomrule
\end{tabular}
\end{table}

\vspace{1em}
\begin{center}
\includegraphics[width=\textwidth]{football_field_valuation.png}
\captionof{figure}{Football-Field Valuation – Ascendra Capital, November 2025}
\end{center}

\clearpage

% ---------- COMPANY OVERVIEW ----------
\section{Company Overview}

\subsection{Business Description}
Universal Health Services (UHS) is one of the largest healthcare management organizations in the U.S., with over 30,000 beds across behavioral health and acute care operations. The company operates:
\begin{itemize}
  \item 342 behavioral health facilities (90 inpatient, 252 outpatient)
  \item 28 acute-care hospitals across 10 states
  \item 30,557 total beds (27,655 owned, 2,902 leased)
\end{itemize}

\subsection{Segment Breakdown (FY 2024)}
\begin{table}[H]\centering
\begin{tabular}{lcccc}
\toprule
Segment & Revenue (\$M) & EBITDA (\$M) & Margin & \% of Revenue \\
\midrule
Behavioral Health & 6,895 & 1,567 & 22.7\% & 43.5\% \\
Acute Care & 8,922 & 1,209 & 13.5\% & 56.5\% \\
\midrule
\textbf{Total} & \textbf{15,817} & \textbf{2,776} & \textbf{17.5\%} & \textbf{100\%} \\
\bottomrule
\end{tabular}
\end{table}

\subsection{Key Attributes}
\begin{itemize}
  \item 90.5\% ownership of facilities (rare among peers)
  \item Diversified exposure across Behavioral (44\%) and Acute (56\%) segments
  \item Controlled by the Miller family (90.5\% voting power)
  \item Consistent free cash flow: \$2.1B OCF / \$1.4B FCF in FY2024
\end{itemize}

\clearpage

% ---------- PART B: SOTP (PRIMARY METHOD) ----------
\section{Sum-of-the-Parts (SOTP) Valuation — Primary Method}

\subsection{Why SOTP Is Appropriate}
The market values UHS at a blended multiple (6.7× EV/EBITDA), which obscures the economics of:
(i) real estate ownership vs.\ leasing and (ii) segment quality differences (Behavioral vs.\ Acute).
A segmented SOTP explicitly values (a) the operating companies (OpCos) on normalized, post-rent EBITDA
and (b) the property companies (PropCos) on NOI at healthcare REIT cap rates.

\subsection{SOTP Architecture}
\[
\text{UHS Value} \;=\;
\underbrace{\text{Behavioral OpCo}}_{\text{EBITDA multiple}}
+\underbrace{\text{Behavioral PropCo}}_{\text{NOI / Cap}}
+\underbrace{\text{Acute OpCo}}_{\text{EBITDA multiple}}
+\underbrace{\text{Acute PropCo}}_{\text{NOI / Cap}}
\]

\subsection{Normalization: Owned vs.\ Leased Distortion}
Reported EBITDA from owned facilities embeds an implicit real-estate return (no rent expense),
while leased facilities incur market rent (lowering EBITDA). To create apples-to-apples operating EBITDA,
we normalize in three steps.

\subsubsection*{Step 1 — Compute EBITDAR (Add Back Actual Rent)}
\[
\text{EBITDAR} \;=\; \text{Reported EBITDA} \;+\; \text{Actual Rent on Leased Facilities}
\]

\begin{table}[H]\centering
\caption*{EBITDAR Reconciliation}
\begin{tabular}{lccc}
\toprule
Segment & Reported EBITDA & {+ Rent Paid} & {= EBITDAR} \\
\midrule
Behavioral & \$1{,}567\,M & \$47\,M & \textbf{\$1{,}614\,M} \\
Acute      & \$1{,}209\,M & \$99\,M & \textbf{\$1{,}307\,M} \\
\midrule
\textbf{Total} & \textbf{\$2{,}776\,M} & \textbf{\$146\,M} & \textbf{\$2{,}921\,M} \\
\bottomrule
\end{tabular}
\end{table}

\textit{Source: UHS 10-K FY2024, Note 14 (Segment Information), page references in data room.}

\subsubsection*{Step 2 — Impute Market Rent on Owned Facilities}
We calculate market rent per bed using actual lease payments, then apply to owned facilities.

\begin{table}[H]\centering
\caption*{Market Rent Calculation}
\begin{tabular}{lccc}
\toprule
Segment & Leased Beds & Actual Rent & Rent per Bed \\
\midrule
Behavioral & 1,656 & \$47M & \$28,382/year \\
Acute & 1,246 & \$99M & \$79,455/year \\
\bottomrule
\end{tabular}
\end{table}

\begin{table}[H]\centering
\caption*{Total Market Rent (Actual + Imputed)}
\begin{tabular}{lcccc}
\toprule
Segment & Owned Beds & Imputed Rent & Actual Rent & Total Market Rent \\
\midrule
Behavioral & 22,465 & \$638M & \$47M & \textbf{\$685M} \\
Acute & 5,190 & \$412M & \$99M & \textbf{\$512M} \\
\midrule
\textbf{Total} & \textbf{27,655} & \textbf{\$1,050M} & \textbf{\$146M} & \textbf{\$1,196M} \\
\bottomrule
\end{tabular}
\end{table}

\subsubsection*{Step 3 — Normalized OpCo EBITDA (Post-Rent)}
\[
\text{Normalized OpCo EBITDA} \;=\; \text{EBITDAR} \;-\; \text{Total Market Rent}
\]

\begin{table}[H]\centering
\caption*{Normalized Operating EBITDA (Post-Rent)}
\begin{tabular}{lccc}
\toprule
Segment & EBITDAR & $-$ Total Rent & \,=\, Normalized OpCo EBITDA \\
\midrule
Behavioral & \$1{,}614\,M & \$685\,M & \textbf{\$930\,M} \\
Acute      & \$1{,}307\,M & \$512\,M & \textbf{\$796\,M} \\
\midrule
\textbf{Total} & \textbf{\$2{,}921\,M} & \textbf{\$1{,}196\,M} & \textbf{\$1{,}725\,M} \\
\bottomrule
\end{tabular}
\end{table}

\subsection{PropCo NOI and Valuation}
Under a triple-net structure, PropCo bears minimal operating expense; therefore NOI = Total Market Rent.

\[
\text{PropCo NOI} \;=\; \text{Total Market Rent}
\qquad\Rightarrow\qquad
\text{Value} \;=\; \frac{\text{NOI}}{\text{Cap Rate}}
\]

\begin{table}[H]\centering
\caption*{PropCo Valuation (6.5\% Cap Rate)}
\begin{tabular}{lcc}
\toprule
Segment & PropCo NOI & Cap-Rate Valuation \\
\midrule
Behavioral PropCo & \$685\,M & \textbf{\$10{,}538\,M} \\
Acute PropCo      & \$512\,M & \textbf{\$7{,}877\,M} \\
\midrule
\textbf{Total}    & \textbf{\$1{,}196\,M} & \textbf{\$18{,}415\,M} \\
\bottomrule
\end{tabular}
\end{table}

\subsection{OpCo Multiples and SOTP Build}

\paragraph{OpCo Multiples Justification:}
\begin{itemize}
  \item \textbf{Behavioral OpCo: 9.0×} — Premium justified by 22.7\% margins, 7-9\% growth, peer ACHC trades at 10.2×
  \item \textbf{Acute OpCo: 6.5×} — In line with HCA (7.8×) and THC (6.0×), reflecting 13.5\% margins
\end{itemize}

\begin{table}[H]\centering
\caption*{SOTP Base Case — Enterprise Value Build}
\begin{tabular}{lcc}
\toprule
Component & Calculation & EV (\$M) \\
\midrule
Behavioral OpCo & \$930M × 9.0× & \textbf{8{,}370} \\
Behavioral PropCo & \$685M ÷ 6.5\% & \textbf{10{,}538} \\
Acute OpCo & \$796M × 6.5× & \textbf{5{,}174} \\
Acute PropCo & \$512M ÷ 6.5\% & \textbf{7{,}877} \\
\midrule
\textbf{Enterprise Value} & & \textbf{31{,}959} \\
Less: Net Debt & & (4{,}379) \\
\textbf{Equity Value} & & \textbf{27{,}580} \\
Shares (MM) & & 63.64 \\
\textbf{Per Share} & & \textbf{\$433} \\
\bottomrule
\end{tabular}
\end{table}

\subsection{SOTP Sensitivity}

\begin{table}[H]\centering
\caption*{SOTP Scenarios — Value per Share}
\begin{tabular}{lcccc}
\toprule
Scenario & Beh Multiple & Acute Multiple & Cap Rate & Per Share \\
\midrule
BEAR & 8.5× & 6.0× & 7.0\% & \textbf{\$399} \\
BASE & 9.0× & 6.5× & 6.5\% & \textbf{\$433} \\
BULL & 9.5× & 7.0× & 6.0\% & \textbf{\$471} \\
\bottomrule
\end{tabular}
\end{table}

\subsection{Key Takeaways}
\begin{enumerate}
  \item \textbf{Normalized EBITDA:} Properly accounting for market rent yields \$930M Behavioral, \$796M Acute OpCo EBITDA
  \item \textbf{PropCo Value:} \$18.4B in real estate at 6.5\% cap rate
  \item \textbf{Behavioral Premium:} Higher margins and growth justify 9.0× vs 6.5× for Acute
  \item \textbf{Upside:} 95\% to current price of \$222
\end{enumerate}

\clearpage

% ---------- INTEGRATED FOOTBALL FIELD ----------
\section{Integrated Valuation Summary}

\subsection{Weighted Football-Field Results}

\begin{table}[H]\centering
\caption*{Valuation Summary by Method}
\begin{tabular}{lcccc}
\toprule
Method & Low (\$) & Base (\$) & High (\$) & Weight \\
\midrule
Sum-of-the-Parts (SOTP) & 399 & 433 & 471 & 30\% \\
Discounted Cash Flow (DCF) & 396 & 477 & 558 & 25\% \\
LBO (Financial Buyer) & 265 & 320 & 385 & 20\% \\
Trading Comps & 193 & 236 & 280 & 10\% \\
Precedent M\&A Transactions & 280 & 367 & 455 & 15\% \\
\midrule
\textbf{Weighted Average} & \textbf{333} & \textbf{392} & \textbf{454} & \textbf{100\%} \\
\bottomrule
\end{tabular}
\end{table}

\paragraph{Outcome.}
\begin{itemize}
  \item The blended \textbf{base case value is \$392/share}, implying \textbf{+76\% upside} to current price of \$222.
  \item SOTP base case of \textbf{\$433/share} represents \textbf{+95\% upside}.
  \item Even the conservative low case (\$333) exceeds current levels by 50\%.
\end{itemize}

\subsection{Investment Thesis Recap}
\begin{enumerate}
  \item \textbf{Market Mispricing:} UHS trades at 6.7× EV/EBITDA vs 7.8× industry median
  \item \textbf{Hidden Real Estate:} 90\% facility ownership worth \$18.4B at 6.5\% cap rate
  \item \textbf{Behavioral Premium:} 22.7\% margins deserve 9.0× vs market's implied 4.8×
  \item \textbf{Multiple Expansion:} Proper segmentation yields 95\% upside to \$433
  \item \textbf{Strong Fundamentals:} \$1.4B FCF, net leverage 1.6×, consistent performance
\end{enumerate}

\subsection{Recommended Offer Range}
\begin{table}[H]\centering
\caption*{Strategic Acquisition Framework}
\begin{tabular}{lccc}
\toprule
Offer Level & Premium vs Spot & vs SOTP & Rationale \\
\midrule
\$375 & +69\% & −13\% & Conservative entry, 30\% premium to weighted avg \\
\$410 & +85\% & −5\% & Meets weighted average, competitive \\
\$450 & +103\% & +4\% & Full SOTP recognition, strategic premium \\
\bottomrule
\end{tabular}
\end{table}

\paragraph{Miller Family Control.}
Given 90.5\% voting power, successful transaction likely requires 70-100\% premium, suggesting \textbf{\$375-\$450/share range}.

\clearpage

\section*{Critical Methodology Note}

\subsection*{Correction from Previous Analysis}
This analysis corrects a significant error in earlier SOTP calculations that \textbf{overstated} PropCo NOI and \textbf{understated} OpCo EBITDA.

\begin{table}[H]\centering
\caption*{Comparison: Previous vs Corrected Methodology}
\begin{tabular}{lccc}
\toprule
Metric & Previous (Error) & Corrected & Impact \\
\midrule
\multicolumn{4}{l}{\textbf{Behavioral Segment:}} \\
\quad OpCo EBITDA & \$607M & \textbf{\$930M} & +53\% \\
\quad PropCo NOI & \$1,007M & \textbf{\$685M} & −32\% \\
\multicolumn{4}{l}{\textbf{Acute Segment:}} \\
\quad OpCo EBITDA & \$329M & \textbf{\$796M} & +142\% \\
\quad PropCo NOI & \$978M & \textbf{\$512M} & −48\% \\
\midrule
\textbf{Base Case Value/Share} & \$530 & \textbf{\$433} & −18\% \\
\bottomrule
\end{tabular}
\end{table}

\paragraph{Root Cause:}
The previous model incorrectly allocated market rent, resulting in PropCo NOI being too high and OpCo EBITDA too low. The \textbf{corrected methodology} properly:
\begin{enumerate}
  \item Calculates rent per bed from actual lease data
  \item Imputes market rent only on owned facilities
  \item Derives normalized OpCo EBITDA as EBITDAR minus total market rent
  \item Assigns total market rent as PropCo NOI
\end{enumerate}

\paragraph{Key Insight:}
While the corrected base case value is lower (\$433 vs \$530), it remains \textbf{highly attractive} with 95\% upside and is now \textbf{mathematically rigorous} and defensible.

\vfill
\begin{center}
\textbf{\Large This valuation uses the corrected methodology}\\[6pt]
\textit{All figures verified against UHS 10-K FY2024}\\[3pt]
November 7, 2025
\end{center}

\clearpage

\section*{Appendices}

\subsection*{Data Sources}
\begin{enumerate}
  \item UHS 10-K FY2024 — Segment financials, facility counts, rent expense
  \item ebitda\_normalization\_detail.csv — Detailed PropCo/OpCo split calculations
  \item Healthcare REIT 10-Ks — Cap rate benchmarking (VTR, PEAK, WELL, OHI, DOC, HR, MPW, SBRA)
  \item Bloomberg, CapitalIQ — Peer trading multiples
  \item Precedent transaction analysis — M\&A benchmarking
\end{enumerate}

\vfill
\begin{center}
  \includegraphics[width=0.3\textwidth]{Assets/logo.png}\\[6pt]
  \textbf{\Large Ascendra Capital}\\
  \vspace{0.5em}
  \textit{Confidential – November 7, 2025 Corrected Valuation}
\end{center}

\end{document}
